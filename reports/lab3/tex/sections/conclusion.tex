\subsection{Summary}
The objective of this laboratory session was to provide experience with the development of a user authentication mechanism. The proposed scenario was a client/server application for printing services. The developed system should authenticate every request to a system user, and execute the requested action only if the client was properly authenticated. Password storage, password transport and password verification were issues to be considered for the design.

The proposed solution authenticated the users on each request (individual request authentication).\\
Password storage was performed on a third entity: a database management system (DBMS). This design decision allowed different applications to use the same users' information.\\
Passwords were not stored in clear but hashed using the SHA-256 algorithm. Salts were added to the passwords before hashing them in order to avoid rainbow table attacks.

As demonstrated in \textit{Section \ref{sec:eval}}, the system succeeded authenticating requests, and they were processed only if the client was able to provide a valid username and password.
\subsection{Future work}
The following tasks were identified as possible future work for making the system more secure:
\begin{itemize}
	\item Encrypt communication between clients and server using Transport Layer Security (TLS).
	\item Use of pepper (a secret application-wide random value appended to a password before hashing it) for securing even more users information.
	\item Enable user enrolment.
\end{itemize}